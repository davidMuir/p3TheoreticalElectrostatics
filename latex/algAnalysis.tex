\documentclass{article}
\usepackage{algorithm}
\usepackage[noend]{algpseudocode}
\usepackage{graphicx}
\newcommand*\Let[2]{\State #1 $\gets$ #2}

\begin{document}

\section{Introduction to Algorithm Analysis}
Want to say why it is important to properly analyse algorithms. Namely that we can use this analysis to design more efficient variants of the algorithm and to compare the algorithm with others that we may wish to use.

\section{Theoretical Analysis}

%TODO All of this
\subsection{Finite Difference}
\begin{algorithm}
    \caption{Finite Difference}
    \label{alg:fd}
    \begin{algorithmic}[1]
        \Function{iteration}{old}
            \Let{mo}{old}
            \Let{newm}{old}
            \Let{n}{widthOf(old)}
            \Let{m}{heightOf(old)}
            \Let{change}{0}
            \For{$x \gets 0 \textrm{ to } n$}
                \For{$y \gets 0 \textrm{ to } m$}
                    \If{$mo_{x,y} \textrm{ not a boundary }$}
                        \Let{$newm_{x,y}$}{$[{mo_{x-1,y}+mo_{x+1,y}+mo_{x,y-1}+mo_{x,y+1}}] / {2}$}
                    \EndIf
                \EndFor
            \EndFor
            \Let{next}{newm}
            \Let{next.error}{change}
            \State \Return{next}
        \EndFunction
        \Function{solve}{}
            \Let{first}{grid values}
            \Let{err}{1000}
            \Let{k}{1}
            \While{Not at desired precision AND not at maximum iterations}
                \Let{n}{$iteration(o)$}
                \Let{err}{$error(n)$}
                \Let{o}{n}
                \Let{iterations}{iterations + 1}
            \EndWhile
            \Let{solution}{n}
            \State \Return{solution}
        \EndFunction
    \end{algorithmic}
\end{algorithm}


\subsection{Fast Finite Difference}
\begin{algorithm}
    \caption{Fast Finite Difference}
    \label{alg:ffd}
    \begin{algorithmic}[1]
        \Function{solve}{}
            \Let{one}{grid values}
            \Let{two}{grid values}
            \Let{*current}{\&one}
            \Let{*alternate}{\&two}
            \Let{n}{widthOf(current)-1}
            \Let{n}{heightOf(current)-1}
            \For{$i \gets 0 \textrm{ to MAX, and not error}$}
                \Let{error}{1}
                \Let{temp}{current}
                \Let{current}{alternate}
                \Let{alternate}{temp}
                \For{$x \gets 1 \textrm{ to } n$}
                    \For{$y \gets 1 \textrm{ to } m$}
                        \If{$current_{x,y} \textrm{ not a boundary }$}
                            \Let{$alternate_{x,y}$}{$[{current_{x-1,y}+current_{x+1,y}+current_{x,y-1}+current_{x,y+1}}] / {2}$}
                        \EndIf
                        \If{$error \neq true$}
                            \Let{error}{$|alternate_{x,y} - current_{x,y}| > precision$}
                        \EndIf
                    \EndFor
                \EndFor
            \EndFor
        \EndFunction
    \end{algorithmic}
\end{algorithm}

There are 7 constant operations, 6 operations dependant on the maximum number of iterations set (MAX) and 6 operations dependant on the number of grid elements and the maximum number of iterations. This gives us a complexity of \[L(6n + 6) + 7\], where L is the number of iterations and n is the grid size. We can reduce this to Big O notation as O(n), giving us an algorithm with linear complexity.
\subsection{Asymmetric Finite Volume}
Stick the pseudocode in here -- maybe not, might need to stick this in the appendix.
Count the number of operations

\section{Experiment}

Actually run all of the algorithms to see what the runtime is like for each and compare each on a line graph. 

Use this to compare the theoretical solutions to the actual runtimes.

And compare each of the algorithms together -- are there situations where one performs better than the other?

\subsection{Worst Case}

Ran each algorithm using Maximum number of iterations of 10000 and a precision of 0 so that the full number of iterations would be run.


\begin{figure*}
    \begin{center}
        \includegraphics*[width = 0.8 \textwidth]{../../fd_ffd_comparison.ps}
    \end{center}
\end{figure*}
\begin{figure*}
    \begin{center}
        \includegraphics*[width = 0.8 \textwidth]{../../afv.ps}
    \end{center}
\end{figure*}
We get a gradient of $5.11*10^{-3}$ for afv, $2.11*10^{-4}$ for fd, and $3.66*10{-5}$ for ffd.

\subsection{Average Case}

Run the experiment with precision at some set value, to see if the rate at which they converge is relatively different between the algorithms.

\end{document}
