\documentclass[aps,twocolumn,pre,nofootinbib,10pt]{revtex4-1}

%\usepackage{auto-pst-pdf}
\usepackage{algcompatible}
\usepackage[noend]{algpseudocode}
\usepackage{graphicx}
\graphicspath{{../../}}
\usepackage{amsmath,amssymb,amsfonts,amsthm}
\usepackage{bbm}
\usepackage{newfloat}
\newcommand*\Let[2]{\State #1 $\gets$ #2}

\DeclareFloatingEnvironment[
    fileext=loa,
    listname=List of Algorithms,
    name=ALGORITHM,
    placement=tbhp,
]{algorithm}

\begin{document}
\section{Algorithm Cost}
\subsection{Theoretical Analysis}

To calculate worst-case performance the number of operations performed over the lifetime of the object is recorded in a recurrence relation, allowing a reasonable approximation of relative performance between the algorithms to be discovered.

The recurrence relations are also reduced to Big O notation to classify complexity in a standard format.

\subsubsection{Finite Difference}
\begin{algorithm}
    \caption{Finite Difference}
    \label{alg:fd}
    \begin{algorithmic}[1]
        \Function{iteration}{old}
            \Let{mo}{old}
            \Let{newm}{old}
            \Let{n}{widthOf(old)}
            \Let{m}{heightOf(old)}
            \Let{change}{0}
            \For{$x \gets 0 \textrm{ to } n$}
                \For{$y \gets 0 \textrm{ to } m$}
                    \If{$mo_{x,y} \textrm{ not a boundary }$}
                        \Let{$newm_{x,y}$}{$[{mo_{x-1,y}+mo_{x+1,y}+mo_{x,y-1}+mo_{x,y+1}}] / {2}$}
                    \EndIf
                    \If{$|newm_{x,y}-mo_{x,y}| > change$}
                        \Let{$change$}{$|newm_{x,y}-mo_{x,y}|$}
                    \EndIf
                \EndFor
            \EndFor
            \Let{next}{newm}
            \Let{next.error}{change}
            \State \Return{next}
        \EndFunction
        \Function{solve}{}
            \Let{first}{grid values}
            \Let{err}{1000}
            \Let{k}{1}
            \While{Not at desired precision AND not at maximum iterations}
                \Let{n}{$iteration(o)$}
                \Let{err}{$error(n)$}
                \Let{o}{n}
                \Let{iterations}{iterations + 1}
            \EndWhile
            \Let{solution}{n}
            \State \Return{solution}
        \EndFunction
    \end{algorithmic}
\end{algorithm}

There are 4 constant operations, 9 operations dependant on the number of iterations set, and 11 operations dependant on the size of the grid and the number of iterations set. This produces the following recurrence relation \[L(11n+9)+4\] With L being the number of iterations set and N being the grid size.

This can be reduced to Big O notation as O(n), showing a linear dependance on input.

\subsubsection{Fast Finite Difference}

The Fast Finite Difference method is an optimisation of the Finite Difference method. It is functionally the same, and arrives at the same approximation as the Finite Difference method, however it takes much less time to do so.

\begin{algorithm}
    \caption{Fast Finite Difference}
    \label{alg:ffd}
    \begin{algorithmic}[1]
        \Function{solve}{}
            \Let{one}{grid values}
            \Let{two}{grid values}
            \Let{*current}{\&one}
            \Let{*alternate}{\&two}
            \Let{n}{widthOf(current)-1}
            \Let{n}{heightOf(current)-1}
            \For{$i \gets 0 \textrm{ to MAX, and not error}$}
                \Let{error}{1}
                \Let{temp}{current}
                \Let{current}{alternate}
                \Let{alternate}{temp}
                \For{$x \gets 1 \textrm{ to } n$}
                    \For{$y \gets 1 \textrm{ to } m$}
                        \If{$current_{x,y} \textrm{ not a boundary }$}
                            \Let{$alternate_{x,y}$}{$[{current_{x-1,y}+current_{x+1,y}+current_{x,y-1}+current_{x,y+1}}] / {2}$}
                            \If{$error \neq true$}
                                \Let{error}{$|alternate_{x,y} - current_{x,y}| > precision$}
                            \EndIf
                        \EndIf
                    \EndFor
                \EndFor
            \EndFor
        \EndFunction
    \end{algorithmic}
\end{algorithm}

There are 7 constant operations, 6 operations dependant on the maximum number of iterations set (MAX) and 6 operations dependant on the number of grid elements and the maximum number of iterations. This produces the following recurrence relation \[L(6n + 6) + 7\]

This can be reduced to O(n), again showing linear dependance on input.

\subsubsection{Asymmetric Finite Volume}

\footnote{Due to the scale of this method, pseudocode has been omitted. The current implementation is available in a public repository should the reader require it.}

There are 11 constant operations, 19 operations dependant on the maximum number of iterations set (MAX) and 34 operations dependant on the number of grid elements and the maximum number of iterations. This produces the following recurrence relation \[L(34N+19)+11\]

This can be reduced to O(n), showing linear dependance on input.

\subsubsection{Aside on Theoretical Treatment of Algorithm Analysis}

The chosen method for analysing algorithm complexity is a standard practise. However, due to differing costs for individual classes of operations only an approximation of complexity may be achieved.

Therefore the previous relations should not be taken as an absolute measure of time taken, but rather as an estimate.

\subsection{Experimental Analysis}
\subsubsection{Worst Case}

To obtain the worst-case performance for the algorithms they were each run iteratively on increasing grid sizes of a constant number of iterations of $10^4$. The run times were recorded using standard library timing methods and a comparison plot (figure \ref{fig:wcalgcomp}) was created.

The experimental set up for this involved creating three identical Grids, with a flow of -50 to 50; a circular conductor set at the mid point of the grid with a radius of a tenth of the height of the grid.

\begin{figure}
    \caption{A comparison of worst case performances of the three algorithms. There are two scales used due to the vastly differing performances of the finite difference methods and the finite volume method.}
    \label{fig:wcalgcomp}
    \begin{center}
        \includegraphics*[angle=-90,width = \columnwidth ]{comparison_wc.ps}
    \end{center}
\end{figure}

A gradient of $5.11 \times 10^{-3}$ is observed for asymmetric finite volume; $2.11 \times 10^{-4}$ for Finite Difference; and $3.66 \times 10^{-5}$ for Fast Finite Difference.

\subsubsection{Average Case}

To obtain the average-case performance for the algorithms they were each run iteratively on increasing grid sizes of with a requested precision of $1 \times 10^{-4}$. The run times were recorded using standard library timing methods and a comparison plot (figure \ref{fig:acalgcomp}) was created.

\begin{figure}
    \caption{A comparison of average case performances of the three algorithms. There are two scales used due to the vastly differing performances of the finite difference methods and the finite volume method}
    \label{fig:acalgcomp}
    \begin{center}
        \includegraphics*[angle=-90,width=\columnwidth]{comparison_ac.ps}
    \end{center}
\end{figure}

After running this we can see that the gradient of the Asymmetric Finite Volume method is $5.16*10^{-3}$, Finite Difference is $2.21*10^{-4}$ and Fast Finite Difference is $3.30*10^{-5}$

A gradient of $5.11 \times 10^{-3}$ is observed for asymmetric finite volume; $2.11 \times 10^{-4}$ for Finite Difference; and $3.66 \times 10^{-5}$

\subsection{Comparison of the methods}

There is a clear trend between the number of operations performed per iteration and the relative performances of the algorithms. With the fastest algorithm, Fast Finite Difference having 6 operations per iteration; Fast Finite Difference having 11 operations per iteration and having a worst case runtime of 5.77 times that of Fast Finite Difference; and Asymmetric Finite Volume having 34 operations per iteration and a runtime of 139.62 of that of Fast Finite Difference.

\end{document}

