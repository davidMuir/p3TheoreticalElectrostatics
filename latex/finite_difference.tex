\documentclass[a4paper, 11pt]{article}
\usepackage[margin=3cm]{geometry}

\begin{document}
\section{Finite Difference Method}

\subsection{Approximation of second order derivative}
Consider the arbitrary function \(f(x)\). The taylor series of this function about the value \(x=a\) is;

\[f(x) = f(a) +(x-a)f^{(1)}(x)+\frac{(x-a)^{2}f^{(2)}(x)}{2!}+\frac{(x-a)^{3}f^{(3)}(x)}{3!}+...\]

Let \(x=x_i + h\) \(a=x_i\) then;

\[f(x_i+h)=f(x_i)+hf^{(1)}(x_i)+\frac{h^2f^{(2)}(x_i)}{2!}+\frac{h^3f^{(3)}(x_i)}{3!}+...\]

Similarly;

\[f(x_i-h)=f(x_i)-hf^{(1)}(x_i)+\frac{h^2f^{(2)}(x_i)}{2!}-\frac{h^3f^{(3)}(x_i)}{3!}+...\]
Therefore;

\[f(x_i+h)+f(x_i-h) = 2f(x_i) + h^2f^{(2)}(x_i) + \frac{h^4}{12}f^{(4)}(x_i)+...\]

If h is very small then only the first two terms are significant and the later terms can be ignored to yield a good approximation.

\[f(x_i+h)+f(x_i-h) \approx 2f(x_i) + h^2f^{(2)}(x_i)\]

So, by rearranging, the second order derivative of a function at \(x_i\) can be approximated by;
\[\frac{d^2f}{dx^2}\Bigg|_{x_i} = \frac{f(x_i+h)-2f(x_i)+f(x_i-h)}{h^2}\]

A similar expression can be found for functions with two variables

\[\frac{\partial^2f}{\partial x^2}\Bigg|_{x_i} = \frac{f(x_i+h,y_j)-2f(x_i,y_j)+f(x_i-h,y_j)}{h^2}\]

Note: the above expression was found by assuming f(x,y) can be expressed as the product of two separate functions, one of x and one of y. \\


\subsection{Laplace Equation}

The Laplace equation is;
\[\nabla^2\phi = 0\]

In two dimensions this can be expressed as;
\[\frac{\partial^2\phi}{\partial x^2} + \frac{\partial^2\phi}{\partial y^2} = 0\]
Let \(x=x_i\) and \(y=y_j\) where \(x_i\) and \(x_j\) are arbitrary. Then;
\[\frac{\partial^2\phi}{\partial x^2}\Bigg|_{(x_{i},y_{j})} + \frac{\partial^2\phi}{\partial y^2}\Bigg|_{(x_{i},y_{j})} = 0\]

Using the results from the previous section this can be written;

\[\frac{\phi(x_i+h,y_j)-2f(x_i,y_j)+\phi(x_i-h,y_j)}{h^2} +  \frac{\phi(x_i,y_j+h)-2f(x_i,y_j)+\phi(x_i,y_j-h)}{h^2}=0\]

Rearranging this gives;

\[\phi(x_i,y_j)=\frac{1}{4}\Bigg(\phi(x_i+h,y_j)+\phi(x_i-h,y_j)+\phi(x_i,y_j+h)+\phi(x_i,y_j-h)\Bigg)\]

Suppose the x,y plane was divided int a grid with lines separated by h in both the x and y direction. et the value of \(\phi\) at an arbitrary point of the grid be denoted by \(\phi_{i,j}\), then the adjacent points in the x and y directions will be denotre \(\phi_{i\pm1,j}\) and \(\phi_{i,j\pm1}\) respectively.\\Therefore;

\[\phi(x_i,y_j)=\frac{1}{4}(\phi(x_{i+1},y_j)+\phi(x_{i-1},y_j)+\phi(x_i,y_{j+1})+\phi(x_i,y_{j-1}))\]

The approximate solution to Laplace's equation at a point, provided the separation of the points is uniform and very small, is the average of the four adjacent points and so it is uite straight forward to numerically solve Laplac'e equation provided there are known boundary conditions.


\end{document}

